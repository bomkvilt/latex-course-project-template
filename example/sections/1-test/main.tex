\section{Часть 1}

\subsection{Формулы}
Вставим формулы.
Это формула 1:
\IncludeEquation{1}[test-1]

Это формула 2:
\IncludeEquation{indir/2}[test-2]

Ещё куча формул:
\IncludeEquation{1}[test-0.1]
\IncludeEquation{1}[test-0.2]
\IncludeEquation{1}[test-0.3]
\IncludeEquation{1}[test-0.4]
\IncludeEquation{1}[test-0.5]

Они добавлены при помощие команды \textcolor{green}{$\backslash includeEquation$}. 
Номер первого уравнения - \req{test-1}, а второго - \req{test-2}. Общая ссылка: \req{test-1, test-2}.
Общая ссылка на большее количество формул: \req{test-1, test-2, test-0.1, test-0.2, test-0.3, test-0.4, 
test-0.5}. Как видно происходит группировка ссылкок в диапазоны.
\pagebreak

\subsection{Картинки}
Теперь вставим изображения. Это фото Дональда Кнута:
\IncludeFigure{knuth1}\bigbreak

Теперь поиграем с его размером и подписью:
\IncludeFigure{knuth1}[knuth1][Вот это подпись под рисунком][height=3cm]\bigbreak
\IncludeFigure{knuth2}[knuth2][Вот это подпись под рисунком][height=3cm]\bigbreak
\IncludeFigure{knuth1}[knuth3][][height=1cm]\bigbreak
\IncludeFigure{knuth2}[knuth4][][height=1cm]\bigbreak

Сошлёмся на вторую пикчу. Её номер - \rfg{knuth1}. Тут ссылка на несколько пикчей: \rfg{knuth1, knuth2}.
И более полная версия: \rfg{knuth1, knuth2, knuth3, knuth4}. Тут тоже производится группировка.
\pagebreak

\subsection{Многофайловость}
Это часть 1.
\IncludeEquation{1}[test-1.1]
Это вот часть вторая, что включает часть 3 по section-relative пути.

\IncludeSubsection{indir/inindir/part3}

