\makeatletter
\addModuleCode
{
    \usepackage{csquotes}
} {

    \usepackage[%
        backend=biber,      % движок
        bibencoding=utf8,   % кодировка bib файла
        sorting=none,       % настройка сортировки списка литературы
        style=gost-numeric, % стиль цитирования и библиографии (по ГОСТ)
        language=autobib,   % получение языка из babel/polyglossia, default: autobib % если ставить autocite или auto, то цитаты в тексте с указанием страницы, получат указание страницы на языке оригинала
        autolang=other,     % многоязычная библиография
        clearlang=true,     % внутренний сброс поля language, если он совпадает с языком из babel/polyglossia
        defernumbers=true,  % нумерация проставляется после двух компиляций, зато позволяет выцеплять библиографию по ключевым словам и нумеровать не из большего списка
        sortcites=true,     % сортировать номера затекстовых ссылок при цитировании (если в квадратных скобках несколько ссылок, то отображаться будут отсортированно, а не абы как)
        doi=false,          % Показывать или нет ссылки на DOI
        isbn=false,         % Показывать или нет ISBN, ISSN, ISRN
        ]{biblatex}[2016/09/17]

    \ltx@iffilelater{biblatex-gost.def}{2017/05/03}%
    {
        \toggletrue{bbx:gostbibliography}%
        \renewcommand*{\revsdnamepunct}{\addcomma}
    } {}


    \providebool{blxmc} % biblatex version needs and has MakeCapital workaround
    \boolfalse  {blxmc} % setting our new boolean flag to default false


    %%% Подключение файлов bib %%%
    \NewDocumentCommand{\setupBiblio}{ m }
    {
        \addbibresource[label=bl-refs]{#1}
    }


    % http://tex.stackexchange.com/a/141831/79756
    % There is a way to automatically map the language field to the langid field. 
    % The following lines in the preamble should be enough to do that.
    % This command will copy the language field into the langid field and will then 
    % delete the contents of the language field. The language field will only be 
    % deleted if it was successfully copied into the langid field.
    \DeclareSourcemap{ % модификация bib файла перед тем, как им займётся biblatex
        \maps{
            \map{% перекидываем значения полей language в поля langid, которыми пользуется biblatex
                \step[fieldsource=language, fieldset=langid, origfieldval, final]
                \step[fieldset=language, null]
            }
            \map{% перекидываем значения полей numpages в поля pagetotal, которыми пользуется biblatex
                \step[fieldsource=numpages, fieldset=pagetotal, origfieldval, final]
                \step[fieldset=numpages, null]
            }
            \map{% перекидываем значения полей pagestotal в поля pagetotal, которыми пользуется biblatex
                \step[fieldsource=pagestotal, fieldset=pagetotal, origfieldval, final]
                \step[fieldset=pagestotal, null]
            }
            \map[overwrite]{% перекидываем значения полей shortjournal, если они есть, в поля journal, которыми пользуется biblatex
                \step[fieldsource=shortjournal, final]
                \step[fieldset=journal, origfieldval]
                \step[fieldset=shortjournal, null]
            }
            \map[overwrite]{% перекидываем значения полей shortbooktitle, если они есть, в поля booktitle, которыми пользуется biblatex
                \step[fieldsource=shortbooktitle, final]
                \step[fieldset=booktitle, origfieldval]
                \step[fieldset=shortbooktitle, null]
            }
            \map{% если в поле medium написано "Электронный ресурс", то устанавливаем поле media, которым пользуется biblatex, в значение eresource.
                \step[fieldsource=medium,
                match=\regexp{Электронный\s+ресурс},
                final]
                \step[fieldset=media, fieldvalue=eresource]
                \step[fieldset=medium, null]
            }
            \map[overwrite]{% стираем значения всех полей issn
                \step[fieldset=issn, null]
            }
            \map[overwrite]{% стираем значения всех полей abstract, поскольку ими не пользуемся, а там бывают "неприятные" латеху символы
                \step[fieldsource=abstract]
                \step[fieldset=abstract,null]
            }
            \map[overwrite]{ % переделка формата записи даты
                \step[fieldsource=urldate,
                match=\regexp{([0-9]{2})\.([0-9]{2})\.([0-9]{4})},
                replace={$3-$2-$1$4}, % $4 вставлен исключительно ради нормальной работы программ подсветки синтаксиса, которые некорректно обрабатывают $ в таких конструкциях
                final]
            }
            \map[overwrite]{ % стираем ключевые слова
                \step[fieldsource=keywords]
                \step[fieldset=keywords,null]
            }
            \map[overwrite,foreach={authorvak,authorscopus,authorwos,authorconf,authorother,authorparent,authorprogram}]{ % записываем информацию о типе публикации в ключевые слова
                \step[fieldsource=$MAPLOOP,final=true]
                \step[fieldset=keywords,fieldvalue={,biblio$MAPLOOP},append=true]
            }
            \map[overwrite]{ % добавляем ключевые слова, чтобы различать источники
                \perdatasource{biblio/external.bib}
                \step[fieldset=keywords, fieldvalue={,biblioexternal},append=true]
            }
            \map[overwrite]{ % добавляем ключевые слова, чтобы различать источники
                \perdatasource{biblio/author.bib}
                \step[fieldset=keywords, fieldvalue={,biblioauthor},append=true]
            }
            \map[overwrite]{ % добавляем ключевые слова, чтобы различать источники
                \perdatasource{biblio/registered.bib}
                \step[fieldset=keywords, fieldvalue={,biblioregistered},append=true]
            }
            \map[overwrite]{ % добавляем ключевые слова, чтобы различать источники
                \step[fieldset=keywords, fieldvalue={,bibliofull},append=true]
            }
            \map[overwrite]{% перекидываем значения полей howpublished в поля organization для типа online
                \step[typesource=online, typetarget=online, final]
                \step[fieldsource=howpublished, fieldset=organization, origfieldval]
                \step[fieldset=howpublished, null]
            }
        }
    }

    \DeclareSourcemap{
        \maps{
            \map{% использование media=text по умолчанию
                \step[fieldset=media, fieldvalue=text]
            }
        }
    }


    %%% Убираем неразрывные пробелы перед двоеточием и точкой с запятой %%%
    \renewcommand*{\addcolondelim}
    {
        \begingroup
        \def\abx@colon
        {
            \ifdim\lastkern>\z@\unkern\fi
            \abx@puncthook{:}\space
        }
        \addcolon
        \endgroup
    }

    \renewcommand*{\addsemicolondelim}
    {
        \begingroup
        \def\abx@semicolon{
            \ifdim\lastkern>\z@\unkern\fi
            \abx@puncthook{;}\space
        }
        \addsemicolon
        \endgroup
    }


    %%% Правка записей типа thesis, чтобы дважды не писался автор
    \DeclareBibliographyDriver{thesis}
    {
        \usebibmacro{bibindex}
        \usebibmacro{begentry}
        \usebibmacro{heading}
        \newunit
        \usebibmacro{author}
        \setunit*{\labelnamepunct}
        \usebibmacro{thesistitle}
        \setunit{\respdelim}
        % \printnames[last-first:full]{author} % Вот эту строчку нужно убрать, чтобы автор диссертации не дублировался
        \newunit\newblock
        \printlist[semicolondelim]{specdata}
        \newunit
        \usebibmacro{institution+location+date}
        \newunit\newblock
        \usebibmacro{chapter+pages}
        \newunit
        \printfield{pagetotal}
        \newunit\newblock
        \usebibmacro{doi+eprint+url+note}
        \newunit\newblock
        \usebibmacro{addendum+pubstate}
        \setunit{\bibpagerefpunct}\newblock
        \usebibmacro{pageref}
        \newunit\newblock
        \usebibmacro{related:init}
        \usebibmacro{related}
        \usebibmacro{finentry}
    }


    %%% Тире как разделитель в библиографии традиционной руской длины:
    \renewcommand*{\newblockpunct}{\addperiod\addnbspace\cyrdash\space\bibsentence}


    %%% В списке литературы обозначение одной буквой диапазона страниц англоязычного источника %%%
    \DefineBibliographyStrings{english}
    {
        pages = {p\adddot} %заглавность буквы затем по месту определяется работой самого biblatex
    }


    %%% Исправление длины тире в диапазонах %%%
    % \cyrdash --- тире «русской» длины, \textendash --- en-dash
    \DefineBibliographyExtras{russian}
    {
        \protected\def\bibrangedash   {\cyrdash\penalty\value{abbrvpenalty}} % almost unbreakable dash
        \protected\def\bibdaterangesep{\bibrangedash} %тире для дат
    }
    \DefineBibliographyExtras{english}
    {
        \protected\def\bibrangedash   {\cyrdash\penalty\value{abbrvpenalty}} % almost unbreakable dash
        \protected\def\bibdaterangesep{\bibrangedash} %тире для дат
    }


    % Set higher penalty for breaking in number, dates and pages ranges
    \setcounter{abbrvpenalty}{10000} % default is \hyphenpenalty which is 12


    % Set higher penalty for breaking in names
    \setcounter{highnamepenalty}{10000} % If you prefer the traditional BibTeX behavior (no linebreaks at highnamepenalty breakpoints), set it to ‘infinite’ (10 000 or higher).
    \setcounter{lownamepenalty}{10000}


    %%% Set low penalties for breaks at uppercase letters and lowercase letters
    \setcounter{biburllcpenalty}{500 } %управляет разрывами ссылок после маленьких букв, RTFM biburllcpenalty
    \setcounter{biburlucpenalty}{3000} %управляет разрывами ссылок после больших   букв, RTFM biburlucpenalty


    %%% Список литературы с красной строки (без висячего отступа) %%%
    \defbibenvironment{bibliography} % переопределяем окружение библиографии из gost-numeric.bbx пакета biblatex-gost
    {   
        \list
        {
            \printtext[labelnumberwidth] 
            {
                \printfield{prefixnumber}
                \printfield{labelnumber}
            }
        }
        {
            \setlength{\labelwidth}{\labelnumberwidth}%
            \setlength{\leftmargin}{0pt}            % default is \labelwidth
            \setlength{\labelsep}{\widthof{\ }}     % Управляет длиной отступа после точки % default is \biblabelsep
            \setlength{\itemsep}{\bibitemsep}       % Управление дополнительным вертикальным разрывом между записями. \bibitemsep по умолчанию соответствует \itemsep списков в документе.
            \setlength{\itemindent}{\bibhang}       % Пользуемся тем, что \bibhang по умолчанию принимает значение \parindent (абзацного отступа), который переназначен в styles.tex
            \addtolength{\itemindent}{\labelwidth}  % Сдвигаем правее на величину номера с точкой
            \addtolength{\itemindent}{\labelsep}    % Сдвигаем ещё правее на отступ после точки
            \setlength{\parsep}{\bibparsep}         %
        }
        \renewcommand*{\makelabel}[1]{\hss##1}
    }
    { \endlist }
    { \item    }


    \defbibheading{nobibheading}{} % пустой заголовок, для подсчёта публикаций с помощью невидимой библиографии
    \defbibheading{pubgroup}{\section*{#1}} % обычный стиль, заголовок-секция
    \defbibheading{pubsubgroup}{\noindent\textbf{#1}} % для подразделов "по типу источника"


    \NewDocumentCommand{\insertBiblio}{}
    {
        \printbibliography[title=Список литературы]
    }
}
\makeatother
