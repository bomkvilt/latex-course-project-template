
\usepackage[no-math]{fontspec}

\usepackage{polyglossia}[2014/05/21]
\usepackage{xecyr}

% set main and additional languages
\setmainlanguage[babelshorthands=true,indentfirst=true]{russian}
\setotherlanguage{english}

\setmonofont{Courier New}                          % моноширинный шрифт
\newfontfamily\cyrillicfonttt{Courier New}         % моноширинный шрифт для кириллицы
\defaultfontfeatures{Ligatures=TeX}                % стандартные лигатуры TeX, замены нескольких дефисов на тире и т. п. Настройки моноширинного шрифта должны идти до этой строки, чтобы при врезках кода программ в коде не применялись лигатуры и замены дефисов
\setmainfont{Times New Roman}                      % Шрифт с засечками
\newfontfamily\cyrillicfont{Times New Roman}       % Шрифт с засечками для кириллицы
\setsansfont{Arial}                                % Шрифт без засечек
\newfontfamily\cyrillicfontsf{Arial}               % Шрифт без засечек для кириллицы
